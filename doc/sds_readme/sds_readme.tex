%        File: sds_readme.tex
%     Created: Thu Jun 07 08:00 PM 2012 C
% Last Change: Thu Jun 07 08:00 PM 2012 C
%
\documentclass[a4paper]{article}
\usepackage{graphicx}
\usepackage{indentfirst}
\usepackage{algorithm}
\usepackage{xcolor}
\usepackage{listings}

\definecolor{dkgreen}{rgb}{0,0.6,0}
\definecolor{gray}{rgb}{0.5,0.5,0.5}
\definecolor{mauve}{rgb}{0.58,0,0.82}

\lstset{ %
  basicstyle=\footnotesize,           % the size of the fonts that are used for the code
  numbers=none,                   % where to put the line-numbers
  numberstyle=\footnotesize,          % the size of the fonts that are used for the line-numbers
  stepnumber=2,                   % the step between two line-numbers. If it's 1, each line
                                  % will be numbered
  numbersep=5pt,                  % how far the line-numbers are from the code
  backgroundcolor=\color{white},      % choose the background color. You must add \usepackage{color}
  showspaces=false,               % show spaces adding particular underscores
  showstringspaces=false,         % underline spaces within strings
  showtabs=false,                 % show tabs within strings adding particular underscores
  frame=none,                   % adds a frame around the code
  tabsize=2,                      % sets default tabsize to 2 spaces
  captionpos=b,                   % sets the caption-position to bottom
  breaklines=true,                % sets automatic line breaking
  breakatwhitespace=false,        % sets if automatic breaks should only happen at whitespace
  title=\lstname,                   % show the filename of files included with \lstinputlisting;
                                  % also try caption instead of title
  numberstyle=\tiny\color{gray},        % line number style
  keywordstyle=\color{blue},          % keyword style
  commentstyle=\color{dkgreen},       % comment style
  stringstyle=\color{mauve},         % string literal style
  escapeinside={\%*}{*)},            % if you want to add a comment within your code
  morekeywords={*,...}               % if you want to add more keywords to the set
}


\begin{document}

\title{Ucore on PIA User Guide}
\author{Chen Yuheng \\ Tsinghua Unv. \and Yang Yang\\ Tsinghua Unv.}
\maketitle

\section{Introduction}
\emph{Ucore} is an experimental operating system that is used in OS course in Tsinghua. Instead of communicating with PC through serial port, Ucore on PIA utilizes a new interface, called \emph{SD Slave (SDS)}, to provide Input/Output service between the card and your PC. Although you
can still enable UART (the serial port), some advanced features, including file transmitting and KGDB debugging, are SDS specific.

This document will only give practical instructions on our system. If you want to know more about how this system works, please see our other documents.

\section{Development Environment}
In order to use the SDS, you must make sure that there is a native SD/MMC reader on your PC (converter is not supported), for example:

\begin{verbatim}
$ lspci -vv | grep sd
    kernel driver in use: sdhci-pci
\end{verbatim}

We have tested our usermode SDS host controller \emph{um\_sds} in the following OS:

\begin{itemize}
\item 32-bit Ubuntu 10.04 LTS, Linux 2.6
\item 64-bit Gentoo, Linux 3.2 and 3.4
\end{itemize}

We also develop a kernel mode driver but is not recommended. Contact us
if you need it.

Other tools you need:
\begin{itemize}
\item Minicom: available in most Linux distributions.
\item ARM9 Toolchain: needed if you want to recompile ucore.
\end{itemize}

\section{Booting}
NOTE: The shipped PIA board contains a precompiled bootloader and a 
ucore kernel, and can boot directly! Please skip to Section 2.
\subsection{Prepare An SD Card}

On the PIA development board, you will need a normal SD card to store
the bootloader(BOOT.BIN) and the kernel(image.bin).
Since the SD card will be formatted, backup your data first.

To figure out the device name of your SD card:
\begin{verbatim}
root@ubuntu:/home/chenyh# fdisk -l
Disk /dev/mmcblk0: 3904 MB, 3904897024 bytes
100 heads, 35 sectors/track, 2179 cylinders
Units = cylinders of 3500 * 512 = 1792000 bytes
Sector size (logical/physical): 512 bytes / 512 bytes
I/O size (minimum/optimal): 512 bytes / 512 bytes
Disk identifier: 0x00000000

        Device Boot      Start         End      Blocks   Id  System
/dev/mmcblk0p1               3        2180     3809280    b  W95 FAT32
\end{verbatim}

Make partitions:
\begin{verbatim}
root@ubuntu:/home/chenyh# fdisk /dev/mmcblk0
//create two partitions
        Device Boot      Start         End      Blocks   Id  System
/dev/mmcblk0p1               1         294      514482+   6  FAT16
/dev/mmcblk0p2             295        2179     3298750   83  Linux
\end{verbatim}

Format partitions:
\begin{verbatim}
root@ubuntu:/home/chenyh# mkfs.vfat -n "PIA_boot" /dev/mmcblk0p1
root@ubuntu:/home/chenyh# mkfs.ext2 -L "PIA_root" /dev/mmcblk0p2
\end{verbatim}

\subsection{Copy the Binaries to Your SD Card}
It is recommended to try the precompiled bootloader and kernel before
you make any changes.

TODO

\subsection{Check the Connection}
Now insert your PIA and check the connection:
\begin{enumerate}
  \item Power up PIA, wait for 5 seconds (SDS initializing)
  \item Insert PIA into the SD slot on your PC
  \item Check your connection:
    \begin{verbatim}
$ dmesg|tail
[ 1835.325238] mmc0: new SDHC card at address 0000
[ 1835.325457] mmcblk0: mmc0:0000 SDS0 3.69 GiB 
[ 1835.330193]  mmcblk0:

$ ls /dev/mmc*
/dev/mmcblk0
    \end{verbatim}

\end{enumerate}

\section{Console Input/Output}

\subsection{Start the Usermode SDS Daemon}

\begin{verbatim}
chenyh@ubuntu:~/os/driver/demo$ sudo ./um_sds 
[sudo] password for chenyh: 
Waiting SDS...
listening @ unix socket: sds.socket...
file >>KGDB listening at PORT 1234
kgdb_wait_target_break: Waiting target...
>>
\end{verbatim}

If you can not see these lines, try to reset the PIA and 
check the SD connection. 

\subsection{Use Minicom to Communicate with PIA}
Now, you can use Minicom to access your console on PIA:
\begin{verbatim}
chenyh@ubuntu:~/os/driver/demo$ minicom -D unix#`pwd`/sds.socket
\end{verbatim}

Now you should be able to get a remote shell:
\begin{verbatim}
cpu_idle: starting
kernel_execve: pid = 2, name = "sh".
Welcome to sh!
$                                   
\end{verbatim}

\subsection{Enable UART}
By default, Ucore only uses SDS as its I/O device. If you need
to enable UART, you have to change the source code of ucore's kernel:

\begin{verbatim}
TODO
\end{verbatim}

And recompile your kernel\footnote{To figure out how to compile
ucore, see \emph{UCore Porting on ARM9}}.

After connect the PIA with your PC through a serial cable, 
reconfigure your Minicom:

\begin{verbatim}
TODO
\end{verbatim}

\section{File Uploading}
Only available in \emph{um\_sds}. The sent file is only stored
in the memory on PIA, you will need two steps to write it to
Ucore's filesystem:
\begin{enumerate}
\item
\begin{verbatim}
TODO
\end{verbatim}

\item
\begin{verbatim}
TODO
\end{verbatim}
\end{enumerate}

\section{KGDB Debugging}
Only available in \emph{um\_sds}. 

\section{Problems?}
Please contact us: \\
Chen Yuheng: chyh1990@163.com\\
Yang Yang: geraint0923@gmail.com

\end{document}


